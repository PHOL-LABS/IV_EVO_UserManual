\chapter{Preparation and installation}\label{ch:preparation}

Set up the workspace before unpacking the indicator. The pogo connector must stay clean and insulated, and the harness should be ready for the required sensor.

\section{Workspace checklist}
\begin{enumerate}
    \item Disconnect the vehicle battery and place the indicator on a non-conductive mat. Keep metal laptop shells and other conductive objects away from the pogo pins.
    \item Test-fit the mounting location so the harness can reach the target sensor without strain. Route cables away from sharp edges or high-temperature engine parts.
    \item Install the socket supplied with the kit and confirm that the programmer can be attached later without removing surrounding trim.
\end{enumerate}

\section{Preparing the sensor harness}
\begin{itemize}
    \item Use the correct connector for the selected preset. The barometer relies on a VAG \texttt{8K0973703} plug, while the boost sensor needs a sealed three-pin connector such as Bosch \texttt{1928403966}.
    \item When fitting a lambda gauge, plan for two oxygen sensors: the stock narrowband unit for the ECU and a Bosch LSU~4.9 connected to a dedicated controller. Feed the controller's analogue output into the indicator harness.
    \item Provide strain relief and insulation for every splice. The pogo connector and sensor leads are not designed to flex while the indicator is powered.
\end{itemize}

\section{Support}
PHOL-LABS Kft offers a one-time assistance session for wiring questions and provides extended consultations on a paid basis when custom sensor adaptations are required. Contact support before powering the system if any wiring step is unclear.
