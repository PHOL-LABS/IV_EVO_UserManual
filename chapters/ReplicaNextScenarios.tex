\chapter{Configuring with the IV-Conf web pack}\label{ch:web-config}

The IV-Conf web pack is the primary tool for configuring IV-Indicators Nano hardware. Open \url{https://phol-labs.com/iv} in a Chromium-based browser with Web Serial support and connect the supplied programmer via USB.

\section{Connecting to the indicator}
\begin{enumerate}
    \item Click the serial port icon and select the entry that corresponds to the programmer.
    \item Open the \texttt{Settings} panel and set the baud rate to 115200.
    \item Press \texttt{Connect MODBUS}. The status banner should confirm the connection without red error indicators.
\end{enumerate}

\section{Preset catalogue}
Multiple presets exist for each indicator type. The voltmeter, for example, includes variants with multi-bar graphics or yellow highlights. Select a preset that matches the physical gauge before uploading settings.

\begin{table}[htbp]
    \centering
    \caption{Example voltmeter presets}
    \label{tab:voltmeter-presets}
    \begin{tabular}{@{} l l @{}}
        \toprule
        Preset name & Description \\
        \midrule
        IV-Volts-Bar-Multi & Voltmeter with a multi-bar display.
        \\
        IV-Volts-Bar-Multi4 & Four-level multi-bar variant.
        \\
        IV-Volts-BarYellow & Bar display with a yellow colour scheme.
        \\
        IV-Volts-Bar & Standard bar display.
        \\
        IV-Volts & Minimal preset without additional graphics.
        \\
        \bottomrule
    \end{tabular}
\end{table}

\section{Uploading and downloading data}
\begin{itemize}
    \item Use the \texttt{Send data} button to push the current preset or manual adjustments to the indicator. The transfer completes automatically within 10--15~seconds.
    \item Click \texttt{Get data} to read the existing configuration from the gauge before editing ranges or colours.
    \item The \texttt{Stop Send Data} control cancels an in-progress upload if you selected the wrong preset or need to alter parameters.
\end{itemize}

Manual tuning allows the gradient start/end hue, backlight intensity, and numeric sensor ranges to be adapted for custom hardware. Confirm that every change matches the installed indicator to avoid misleading readings.

\section{Interface overview}
\begin{figure}[htbp]
    \centering
    \begin{subfigure}{0.3\textwidth}
        \centering
        \includegraphics[width=\textwidth]{image15.png}
        \caption{Launching the web pack}
    \end{subfigure}\hfill
    \begin{subfigure}{0.3\textwidth}
        \centering
        \includegraphics[width=\textwidth]{image11.png}
        \caption{Selecting the serial port}
    \end{subfigure}\hfill
    \begin{subfigure}{0.3\textwidth}
        \centering
        \includegraphics[width=\textwidth]{image1.png}
        \caption{Granting port access}
    \end{subfigure}
    \par\medskip
    \begin{subfigure}{0.3\textwidth}
        \centering
        \includegraphics[width=\textwidth]{image8.png}
        \caption{Setting the baud rate}
    \end{subfigure}\hfill
    \begin{subfigure}{0.3\textwidth}
        \centering
        \includegraphics[width=\textwidth]{image22.png}
        \caption{Connecting over MODBUS}
    \end{subfigure}\hfill
    \begin{subfigure}{0.3\textwidth}
        \centering
        \includegraphics[width=\textwidth]{image5.png}
        \caption{Preset selector}
    \end{subfigure}
    \par\medskip
    \begin{subfigure}{0.3\textwidth}
        \centering
        \includegraphics[width=\textwidth]{image3.png}
        \caption{Manual tuning controls}
    \end{subfigure}\hfill
    \begin{subfigure}{0.3\textwidth}
        \centering
        \includegraphics[width=\textwidth]{image2.png}
        \caption{Send data confirmation}
    \end{subfigure}
    \caption{Key IV-Conf web pack panels used for day-to-day configuration.}
\end{figure}

\section{Usage notes}
In most situations only the configuration panels are required. Firmware updates are handled separately in \autoref{ch:appendix}. When a setting fails to apply, disconnect and reconnect the MODBUS session and ensure that the correct serial port and baud rate are selected.
